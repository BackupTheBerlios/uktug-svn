\documentclass[a5paper,twoside]{article}

\usepackage{hyperref}           % For hyperlinks
\usepackage[ansinew]{inputenc}  % For "�"
\usepackage{marvosym}   % For currency symbols
\usepackage{eurosym}    % For currency symbols
\usepackage{cwpuzzle}   % For the crossword
\usepackage{rotating}   % For the inverted crossword answers
\usepackage{textcomp}   % For currency symbols
%\usepackage{mathdesign}  % Tried this, it has more currency symbols, but completely changed the default font
\usepackage{cyrillic}   % For the cyrillic letter used for the Serbian dinar
\usepackage{booktabs}   % For tables with TOPRULE, MIDRULE and BOTTOMRULE


% -----------------------------------------------------------------------------------
% Baskerville Definitons
\newcommand{\BK}{\textit{Baskerville}}
\newcommand{\ISSUEDATE}{April 2009}

\hyphenation{Bas-ker-ville}

% -----------------------------------------------------------------------------------
% Definitions for the Constitution
\newcounter{refa}
\newcounter{refb}
\font\manual=manfnt
\newcommand{\MF}{{\manual META}\-{\manual FONT}}
% -----------------------------------------------------------------------------------

\voffset    -2cm
\textheight 16cm

% -----------------------------------------------------------------------------------
% Heading for pages after the first page.
\usepackage{fancyhdr}
\pagestyle{fancy}

% One heading per page, to the margin
\fancyhead[RE]{}
\fancyhead[LE]{\textit{Baskerville}}          % Recto / Odd
\fancyhead[RO]{\textit{Volume 10, Number 1}}  % Verso / Even
\fancyhead[LO]{}

% \lhead{\textit{Baskerville}}
% \rhead{\textit{Volume 10, Number 1}}

\cfoot{-\thepage-}
% -----------------------------------------------------------------------------------

% ============================================================================================
\begin{document}

% -----------------------------------------------------------------------------------
\begin{titlepage}
\begin{center}

\vspace{2cm}

\includegraphics[width=0.95\textwidth]{lion.png} \\ [2cm]

\textsc{\large The Annals of the UK-\TeX Users' Group} \\[0.2cm]
(\href{http://uk.tug.org/}{http://uk.tug.org/}) \\[0.2cm]
{\large ISSN 1354-5930}
 
\vfill

% Bottom of the page
\begin{tabular*}{0.9\textwidth}{@{\extracolsep{\fill}} l c r }
           & Edited by                 &             \\
Vol. 10.1  & Jonathan \textsc{Webley}  & \ISSUEDATE  \\
\end{tabular*}
\end{center}
\end{titlepage}

% -----------------------------------------------------------------------------------
\setcounter{tocdepth}{2}  % Sections only
\tableofcontents

% -----------------------------------------------------------------------------------
\vspace{0.5cm}
\noindent All contributions should be sent to \href{mailto:baskerville@uk.tug.org}{baskerville@uk.tug.org}. Articles on any area of \TeX, its friends, UK-TUG or related topics are very welcome: the Committee is particularly keen to publish articles with a UK \textit{flavour}. Send in your comments on this issue; your suggestions, letters, thoughts, tips and hints, articles, jokes, questions, requests for help, jobs � anything relevant will be considered for publication.

% The deadline for the next issue is [\textbf{DATE}]. 

% ----------------------------------------------------------------------------------------
\vspace{0.5cm}
\noindent \textbf{UK-TUG Committee 2009}

\begin{itemize}
   \setlength{\parskip}{0pt} % Scrunch up to fit on page
   \item Jonathan Fine (Chair)
   \item David Crossland (Secretary and Webmaster)
   \item David Saunders (Treasurer)
   \item Joseph Wright (Membership Secretary)
   \item John Trapp (Training Officer)
   \item Jonathan Underwood
   \item Charles Goldie
   \item Simon Dales
   \item Jonathan Webley (Baskerville Editor, co-opted)
\end{itemize}
The  committee can be contacted at
\href{mailto:uktug-committee@uk.tug.org}{uktug-committee@uk.tug.org}.

% ----------------------------------------------------------------------------------------
\section{Editorial}
Welcome to the revived \BK. \BK\ was for many years the journal of UK-TUG and archived copies are available through our web-site. The UK-TUG committee have decided to start anew publication of \BK\ and I have been co-opted onto the committee as editor. This is the first new issue. Since the purpose of \TeX\ is to produce ``marks on paper'' \BK\ has been printed and posted out; though all issues will also be available on the web.

The journal is named after a serif typeface designed in 1757 by John Baskerville. Previous issues of \BK\ used the Baskerville font but this issue uses the default \textit{Computer Modern} font.

This particular issue is somewhat light in content. The quality of future issues will depend on \textit{you}, the membership of UK-TUG. I do not intent to create all the content myself, and welcome contributions on matters relevant to \TeX\ or UK-TUG.

I think that a newsletter is central to the well-being of a user group such as ours, and I look forwards to hearing your comments on this issue.

\hfill Jonathan Webley

% ----------------------------------------------------------------------------------------
\section{Events}
\subsection{Euro\TeX\ 2009}
Euro\TeX\ 2009 takes place this year in the Hague, the Netherlands, on 31 August through 4 September 2009, and the conference will focus on educational uses of \TeX, such as manuals, presentations and teaching materials. The conference will be in English.

The fee for UK-TUG members is \euro350, which includes everything except the excursion day (which costs \euro75). In particular it includes accommodation and meals.

The official website is \href{http://www.ntg.nl/EuroTeX2009/index.html}{www.ntg.nl/EuroTeX2009}.

\subsection{Bacho\TeX\ 2009}
Bacho\TeX\ 2009 is the XVIIth Polish \TeX\ Users Group Conference.

As usual it will be held at the traditional \TeX ies' and GUST meeting place, Bachotek near Brodnica, in the north-east of Poland, from 29 April until 3 May 2009 inclusive.

The conference aims to get a glimpse of the future, and the title is:
\begin{center} ``\textit{\TeX: at a turning point, or at the crossroads?}'' \end{center}

The community is putting a lot of effort and thought into possible strategies for promoting and developing \TeX\ and related products for the foreseeable future, including:
\begin{itemize}
\item \TeX\, and \TeX-based engines
\item enhanced graphics engines
\item new or improved macro packages
\item user interfaces
\item new fonts
\end{itemize}

Work progresses in many different directions, and thus there is clearly hope for an ever better future, even though the more pessimistic may wonder whether \TeX\ represents an evolutionary dead-end.

In this context the feedback between users and developers of new tools and engines is immensely important -- it might decide whether within a few years we will be looking back at today as a successful turning point or as a bad decision at the crossroads of \TeX's history.

Conference themes or mottos often have little impact on submissions.
% Some say that they have never believed that conference themes or mottos have any impact on submissions anyway. 
The organisers of Bacho\TeX\ think otherwise. There is no restriction on the content of presentations but those that are user-centric and oriented towards the future of \TeX\ with special emphasis on the needs, hopes and dangers are preferred.

Proposals (abstracts) should be e-mailed to the Program Committee: \href{mailto:papers-2009@gust.org.pl}{papers-2009@gust.org.pl}. Bogus\l{}aw Jackowski has been appointmed chairman.

Especially welcome are proposals for \TeX-related tutorials or introductions. Contact the organisers if you have suggestions for tutorials or workshops by others than yourself or about specific topics.

The deadline for abstracts and other proposals is 8 March. The deadline for final papers will soon be published at the conference web site: \href{http://www.gust.org.pl/bachotex/2009/main-en.html?set_language=en}{www.gust.org.pl/BachoTeX/2009}

Exhausted \TeX ies will get a chance to recover their intellectual powers during nightly musical sessions, usually at a bonfire. Participate by bringing your own instruments (and voices)! Nature fans will have the opportunity to enjoy the unspoiled features of the Bachotek lake and surrounding woods. The conference could also be a family event -- the conference site is an enclosed area with a safe and attractive playing ground for children and parents alike.

\vspace{2em}
\noindent \textbf{Call for \TeX\ Pearls} \\
The organisers are seeking to continue the tradition of ``The Pearls of \TeX\ Programming''.
Here, briefly, is what is wanted:
\begin{itemize}
\item short \TeX, MF or MP macro(s)
\item results must be useful, and the solution not obvious
\item easy to explain:  10 minutes at most
\end{itemize}

If you have something that fits the bill, please consider submitting a proposal. If you know of somebody's work that does the same, please let us know, and we will contact that person.  Other details and previously collected Pearls can be found at 
\href{http://www.gust.org.pl/projects/pearls/}{www.gust.org.pl/projects/pearls}

\newpage
\subsection{TUG 2009}
TUG 2009 will take place in Notre Dame, Indiana, from 28-31 July. See 
\href{http://tug.org/tug2009}{http://tug.org/tug2009}
for the registration form, maps, the proposals already accepted, and more.

April has several deadlines related to the conference:

\begin{itemize}
	\item 13 April 2009: This is the deadline for abstract submissions; see
  \href{http://tug.org/tug2009/cfp.html}{http://tug.org/tug2009/cfp.html} for the call for papers.
  
  Although proposals may be accepted after the deadline, of
  course potential attendees would like to know what they'll be seeing.
  So if you'd like to give a talk, please try to submit an abstract by
  the 13th.
  
  \item 17 April 2009: This is the deadline for bursary applications; see
  \href{http://tug.org/bursary}{http://tug.org/bursary} for information and the application form.  No late applications will be accepted.
  
  \item April 27: This is the deadline for the early bird registration discount. After this date, the registration fee will be increased. Register for the conference through Notre Dame's web-site via this link \href{http://tug.org/tug2009/register.html}{http://tug.org/tug2009/register.html}
\end{itemize}


% ----------------------------------------------------------------------------------------
\newpage
\section{The Hound}
This is a somewhat easy, cryptic crossword and the solution can be found later in this issue. \\

\begin{Puzzle}{9}{9}
|[1]C |A |[2]S |E |* |[3]A |[4]W |L |[5]S |.
|U |* |P |* |* |* |O |* |E |.
|[6]S |U |E |* |[7]S |W |E |D |E |.
|P |* |C |* |H |* |B |* |D |.
|* |[8]S |T |E |A |M |E |R |* |.
|[9]B |* |A |* |D |* |G |* |[10]H |.
|[11]U |N |C |L |E |* |[12]O |R |E |.
|G |* |L |* |* |* |N |* |R |.
|[13]S |L |E |D |* |[14]M |E |M |E |.
\end{Puzzle}

\noindent
\begin{tabular*}{\textwidth}{r p{3.69cm} c r p{3.69cm} }
 \multicolumn{2}{l}{\textbf{Across}}    & & \multicolumn{2}{l}{\textbf{Down}} \\
 \textbf{1}  & In Africa, see a         & & \textbf{1}  & In the discus, perhaps, \\
             & container. (4)           & &             & achieve one's peak. (4) \\
 \textbf{3}  & These tools are a        & & \textbf{2}  & It's a sight, the\\
             & product of bad laws. (4) & &             & centilitres in awful  \\
 \textbf{6}  & Misuse this girl. (3)    & &             & cat's pee. (9) \\
 \textbf{7}  & These weeds for veg. (5) & & \textbf{4}  & Beg and owe with one, \\
 \textbf{8}  & On this ship, the wicked & &             & sadly together we  \\
             & queen mates. (7)         & &             & looked dismal. (9) \\
 \textbf{11} & Pawnbroker is mostly     & & \textbf{5}  & Kent's editor knows  \\
             & unclean. (5)             & &             & the issue. (4) \\
 \textbf{12} & Mineral found in         & & \textbf{7}  & Hades loses a ghost. (5) \\
             & store? (3)               & & \textbf{9}  & These insects cause \\
 \textbf{13} & Poor deals are without   & &             & errors. (4) \\
             & a toboggan. (4)          & & \textbf{10} & Sounds like I hear when \\
 \textbf{14} & Idea came from me,       & &             & present. (4) \\
             & twice. (4) \\
\end{tabular*}

% ----------------------------------------------------------------------------------------
% \section{\LaTeX\ Hints \& Tips}
% \textbf{Currency Symbols} \\

\section{Currency Symbols in \LaTeX}
Standard keyboards contain the dollar sign (\$), which, of course, is a special symbol in \TeX, so needs to be prefaced with a backslash or oblique: \textbackslash \$. This symbol works properly in both text mode and math mode.

Keyboards also have a pound sign (�), the use of which requires the package \texttt{inputenc}. Additionally, there is a standard command \textbackslash \texttt{pounds}, which renders as \pounds. This symbol works properly in both text mode and math mode.

Additionally, standard \LaTeX\ contains two commands for these signs:

\textbackslash \texttt{textdollar} which renders as \textdollar, and

\textbackslash \texttt{textsterling} which renders as \textsterling.

\medskip

The euro has its own package, \texttt{eurosym}, which contains these commands:
\begin{center}
\begin{tabular}{ l l l }
\toprule
\textbf{Symbol} & \textbf{\LaTeX}  \\
\midrule
\geneuro        & \textbackslash \texttt{geneuro} \\
\geneuronarrow  & \textbackslash \texttt{geneuronarrow} \\
\geneurowide    & \textbackslash \texttt{geneurowide} \\
\officialeuro   & \textbackslash \texttt{officialeuro} \\
\bottomrule
\end{tabular}
\end{center}
All of these symbols are generated using the ``C'' character of the current body font. The package also contains the command \textbackslash \texttt{euro} which maps to \textbackslash \texttt{officialeuro} but can be altered using a package option.

% --------------------------------------------------------
% Change footnote marker to symbols instead of numbers.
%\renewcommand{\thefootnote}{\fnsymbol{footnote}}
% --------------------------------------------------------

The \texttt{textcomp} package includes these symbols: 

\medskip

% I'm not using "xtab" since each currency requires 2 rows and a rule so I found it easier
% just to control the page breaks manually.

\begin{center}
\begin{tabular}{p{2.5cm} p{6cm}}
\toprule
\textbf{Symbol} & \textbf{\LaTeX}  \\
\textbf{Name}   & \textbf{Used in} \\
\midrule
\textbaht & \textbackslash \texttt{textbaht} \\
baht & Thailand (THB) \\
\midrule
\textcent & \textbackslash \texttt{textcent} \\
cent & US, Canada \\
\midrule
\textcentoldstyle & \textbackslash \texttt{textcentoldstyle} \\
cent, old style & \\
\bottomrule
\end{tabular}

\vfill
\newpage

\begin{tabular}{p{2.5cm} p{6cm}}
\toprule
\textbf{Symbol} & \textbf{\LaTeX}  \\
\textbf{Name}   & \textbf{Used in} \\
\midrule
\textcolonmonetary & \textbackslash \texttt{textcolonmonetary} \\
col\'on  & Costa Rica (CRC), El Salvador (SVC), \\
cedi     & Ghana (GHS) \\
\midrule
\textcurrency & \textbackslash \texttt{textcurrency} \\
         & Generic currency sign, used when no \\
         & other sign is available. \\
\midrule
\textdollaroldstyle & \textbackslash \texttt{textdollaroldstyle} \\
escudo\footnotemark[1] & formerly Portugal (PTE), \\
                       & Cape Verde (CVE) \\
\midrule
\textdong & \textbackslash \texttt{textdong} \\
dong & Vietnam (VND) \\
\midrule
\texteuro & \textbackslash \texttt{texteuro} \\
euro & Eurozone (EUR) \\
\midrule
\textflorin & \textbackslash \texttt{textflorin} \\
florin & Aruba (AWG), \\
       & Netherlands Antilles (ANG) \\
\midrule
\textguarani & \textbackslash \texttt{textguarani} \\
guarani & Paraguay (PYG) \\
\midrule
\textlira & \textbackslash \texttt{textlira} \\
lira & Formerly Italy (ITL) and others \\
\midrule
\textnaira & \textbackslash \texttt{textnaira} \\
naira & Nigeria (NGN) \\
\midrule
\textpeso & \textbackslash \texttt{textpeso} \\
peso & Philippines (PHP) \\
\midrule
\textwon & \textbackslash \texttt{textwon} \\
won & South Korea (KRW), \\
    & North Korea (KPW) \\
\bottomrule
\end{tabular}

\vfill

\footnotetext[1]{This version of the dollar sign with two vertical lines is called the cifr�o. Amounts are generally written so that it serves as the decimal separator, such as 20\textdollaroldstyle00 for 20 escudos.}

\begin{tabular}{p{2.5cm} p{6cm}}
\toprule
\textbf{Symbol} & \textbf{\LaTeX}  \\
\textbf{Name}   & \textbf{Used in} \\
\midrule
\textyen & \textbackslash \texttt{textyen} \\
yen, yuan  & Japan (JPY), China (CNY) \\
\bottomrule
\end{tabular}
\end{center}

\medskip

The \texttt{mathdesign} package redefines \textbackslash \texttt{texteuro} to be compatible with these fonts: \textit{Utopia}, \textit{Charter} or \textit{Garamond}.

\medskip

And then there is the \texttt{marvosym} package which has these symbols:

\medskip

\begin{center}
\begin{tabular}{ l l l }
\toprule
\textbf{Symbol} & \textbf{\LaTeX}  \\
                & \textbf{Use} \\
\midrule
\Denarius   & \textbackslash \texttt{Denarius}\footnotemark[2] \\
            & Obsolete symbol for the German pfennig \\
\EUR        & \textbackslash \texttt{EUR} \\
\EURcr      & \textbackslash \texttt{EURcr} \\
            & Euro compatible with \textit{Courier}. \\
\EURdig     & \textbackslash \texttt{EURdig} \\
            & Euro compatible with \texttt{marvosym} digits. \\
\EURhv      & \textbackslash \texttt{EURhv} \\
            & Euro compatible with \textit{Helvetica}. \\
\EURtm      & \textbackslash \texttt{EURtm} \\
            & Euro compatible with \textit{Times} \textit{Roman}. \\
\EyesDollar & \textbackslash \texttt{EyesDollar} \\
\Shilling   & \textbackslash \texttt{Shilling}\footnotemark[3] \\
\bottomrule
\end{tabular}
\end{center}

In conclusion, \LaTeX\ caters for all common, and some not so common, currency symbols. Unicode, however, has a few additional ones, and is far better documented.

\hfill Jonathan Webley

\vfill

\footnotetext[2]{The denarius was a Roman coin. The dinar is a descendant of the denarius and is used, or was formerly used, by several countries. However, Serbia, for example, use the cyrillic De (\textcyrrm{D}) letter for the dinar.}
\footnotetext[3]{This symbol resembles a beta ($\beta$) but I belive it to be more akin to the German Eszett (\ss). It is possibly a symbol for the schilling, the pre-euro currency of Austria.}

% ----------------------------------------------------------------------------------------
\section{The Hound Answers}
\noindent \textbf{Across} \\
\begin{turn}{180} 
 7. swede,
 8. steamer,
11. uncle,
12. ore,
13. sled,
14. meme
\end{turn} \\
\begin{turn}{180}
 1. case,
 3. awls,
 6. Sue,
\end{turn} \\

\noindent \textbf{Down} \\
\begin{turn}{180}
4. woebegone,
5. seed, 7. shade, 9. bugs, 10. here
\end{turn} \\
\begin{turn}{180}
1. cusp, 2. spectacle, 
\end{turn}

\end{document}
