\section{\pkg{siunitx}: A \LaTeX\ Swiss army knife for units}
\begin{bibunit}
\AUTHOR{Joseph Wright}

%\begin{abstract}
\subsection{Introduction}
The \pkg{siunitx} package is a complete system for typesetting
units in \LaTeX.  It provides a large number of settings,
allowing the user complete control of the output for a fixed
input.  The package grew out of \pkg{SIunits} and
\pkg{SIstyle}, but covers many additional areas.  This article
provides a brief overview of why the new package was written
and some of its key features.
%\end{abstract}

\subsection{Background}

In November 2007, a seemingly simple query about a bug in the
\pkg{SIunits} package \cite{Heldorn2007} was posted to
\texttt{comp.text.tex}. I suggested a bug fix, but on
contacting the author of \pkg{SIunits} found he had no time for
further maintenance.  I therefore found myself somewhat
accidentally as the new package maintainer.  After fixing the
bug at hand, I decided to have a good look over the code, and
to ask for suggestions for improvements.  It soon became very
clear that unit support in \LaTeX\ needed serious attention,
and that tweaks to \pkg{SIunits} would not be enough.  At that
point, \pkg{siunitx} was born.

\subsection{Units in print}

The need for packages to control printing units is not
immediately obvious.  However, the existence of \pkg{SIunits},
\pkg{SIstyle} \cite{Els2006}, \pkg{fancyunits}
\cite{Bauke2007}, \pkg{unitsdef} \cite{Happel2005} and
\pkg{units} \cite{Reichert1998} attests to the desire to go
beyond direct input of units (and values).

Units are important, and without clear rules problems almost
automatically arise.  There are therefore internationally
agreed units: the \emph{Syst\`eme International d'Unit\'es}
\cite{BIPM2008}, usually referred to as SI units.  To go with
the agreed units are a set of rules on how to print both units
and the associated values; the National Institute of Standards and Technology (\textsc{nist}) have a good set of
guidelines for authors \cite{NIST2008}.  Doing this properly
and consistently is much better achieved using pre-build
macros, rather than checking every single use by hand.

At the same time, different publishers have their own conventions.
These do not always follow the `official' rules, and authors
do not want to change their source for every different use.
This again makes the strong case for using adjustable macros
for writing units; changing the style is then only a question
of altering the definition.

Finally, values as well as units have rules, and these rules
vary depending on publisher and country.  The \pkg{numprint}
package \cite{Harders2008} provides a range of tools to
automatically format numbers, to match the desired output
style.  It also includes basic abilities to include units with
these numbers.

\subsection{Enter \pkg{siunitx}}

Given the existing range of packages, the question arises `why
another package?'  The answer is that, although there are
several other packages, none covers everything. For example,
\pkg{SIunits} provides macros such as \cs{metre} to maintain
consistency, but is poor on control of numbers, whereas
\pkg{SIstyle} is stronger on numbers but does not provide unit
macros.  So a combined package, which can do everything, seemed
desirable.  Many of the ideas that I and others had were also
well beyond anything available in the existing solutions.  That
said, a lot of the internals of \pkg{siunitx} is taken
more-or-less verbatim from the existing packages.  The
\pkg{numprint} system has been extensively recycled here; most
of the number-interpreting system is based on the
\pkg{numprint} method.  Indeed, while the initial aim of the
new package was to deal primarily with \emph{units}, much of
the work I have undertaken has been concerned with
\emph{numbers}.

The design of \pkg{siunitx} is intended to be flexible.  Almost
everything is made available as an option, and given the number
of options, key--value controls were the only way to do this.
The package documentation covers (hopefully) all of them, but a
few examples here will illustrate the ideas.

\subsubsection{Some basics}

The basic macro of the package is \cs{SI}.  This takes a unit
and a value, and typesets them together.  The font used for
this is controlled, and the space between them cannot be
broken.  Thus for example \verb|\SI{10}{\metre}| gives
`\SI{10}{\metre}'.  The second argument is a unit, which can
be given as a series of macros, or can be literal text:
\verb|\SI{2}{N.m^2}| gives `\SI{2}{N.m^2}'.  Here, there is
no need to instruct the package to use maths mode for
superscripts: this is handled automatically.

Both the number and unit parts of the input can be used
independently, with macros \cs{num} and \cs{si}, respectively.
The \cs{num} macro can interpret and format a wide range of
numerical input.  So \verb|\num{1e2,3}| gives `\num{1e2,3}'
(when using U.K.~settings, at least).  As with the rest of the
package, the way that numbers are interpreted can be adjusted
by setting the appropriate package options.  For angles, the
\cs{ang} macro is available; it can work with angles as decimal
and arcs: \verb|\ang{1.23}| and  \verb|\ang{1;2;3}| give
`\ang{1.23}' and `\ang{1;2;3}', respectively.

Package options can be set in three ways.  As with most
packages, load-time options are recognised.  There is also a
\cs{sisetup} macro, which adjusts settings for all subsequent
input.  Finally, all of the user macros accept an optional
first argument; this allows changes for a specific item only.

\subsubsection{The unit processor}

One of the new ideas in \pkg{siunitx} is the unit processor.
This takes macro-based units, and can reformat them depending
on the desired output.  `Out of the box',
\verb|\SI{30}{\metre\per\second}| gives
`\SI{30}{\metre\per\second}', whereas if we set the option
\opt{per=slash}, the same input gives
`\SI[per=slash]{30}{\metre\per\second}'. This can be extended
further, allowing the interpretation of powers, including
reciprocals.  Thus, given the input
\verb|\SI{20}{\per\Square\second}|, the result can be
`\SI{20}{\per\Square\second}',
`\SI[per=slash]{20}{\per\Square\second}',
`\SI[per=frac]{20}{\per\Square\second}' or
`\SI[per=frac,fraction=nice]{20}{\per\Square\second}',
depending on the options set.

\subsubsection{Special effects}

A lot of work has gone into making more complex ideas available
for users of \pkg{siunitx}.  Thus the idea of `repeated'
units is easily handled:
\begin{verbatim}
\SI{1 x 2 x 3}{\metre}
\end{verbatim}
yields `\SI{1x2x3}{\metre}', for example.  Setting option
\opt{repeatunits=false}, the alternative (and not entirely
desirable) `\SI[repeatunits=false]{1x2x3}{\metre}' results.

Similarly, the conversion of angles between decimal and
degree--minute--second format is possible.  So
\verb|\ang{1.23}| normally gives the expected `\ang{1.23}';
setting \opt{angformat=arc} gives the alternative form
`\ang[angformat=arc]{1.23}'.

In many parts of the natural sciences, errors in physical
measurements are important.  Two ways are common for showing
these, as a separate error part, and as a bracketed error in
the last digit: \num[seperr]{1.23(4)} and \num{1.24(4)},
respectively.  Using \pkg{siunitx}, both forms can result from
the same input: \verb|\num{1.24(4)}|.  The separated form is
obtained by setting the \opt{seperr} flag.  This works with
exponents and units, for example:
\begin{verbatim}
\SI[seperr]{1.23(4)e5}{\candela}
\end{verbatim}
gives `\SI[seperr]{1.23(4)e5}{\candela}'.  Notice the
automatic addition of brackets to prevent ambiguity: this is an
adjustable package option.

The final extra to highlight is the provision of a new column
for tabular environments, the \texttt{S} column.  This brings
the ability to use the \cs{num} macro to tables, but also
brings control of alignment (including exponents).  A short
example shows the effect:
\begin{verbatim}
\begin{table}
  \centering
  \begin{tabular}{cS[tabformat=3.1]}
    \toprule
    {Entry} & {Distance/\si{\metre}} \\
    \midrule
    1 & 102.3 \\
    2 & 123.2 \\
    3 &   2,3 \\
    4 & 145.  \\
    5 &   2.3 \\
    \bottomrule
  \end{tabular}
\end{table}
\end{verbatim}
\begin{table}
  \centering
  \begin{tabular}{cS[tabformat=3.1]}
    \toprule
    {Entry} & {Distance/\si{\metre}} \\
    \midrule
    1 & 102.3 \\
    2 & 123.2 \\
    3 &   2,3 \\
    4 & 145.  \\
    5 &   2.3 \\
    \bottomrule
  \end{tabular}
\end{table}
Here, the contents of the \texttt{S} column are centred under
the column heading.  Space is reserved for three digits before
the decimal point, and one digit after.  Thus, while the
decimal points are aligned they are \emph{not} at the centre of
the column.  A range of options are provided to allow control
of positioning in \texttt{S} columns.

\subsubsection{Emulation}

Replacing the existing packages raises the issue of users
moving existing manuscripts to \pkg{siunitx}.  To aid this
transition, emulation modes are available for all of the
existing solutions.  This means that users of the existing
packages can experiment with \pkg{siunitx} without needing to
learn new default settings or input conventions.

\subsection{Conclusions}

This short article can only provide a flavour of the abilities
of \pkg{siunitx}.  However, it hopefully highlights the
possibilities made available by the new package.  The package
documentation includes a full range of examples, with almost
all of the package options illustrated.  I hope that I have
covered a wide range of needs, and that \pkg{siunitx} will make
it easier for \LaTeX\ users to get units right without working
too hard.  Suggestions for new features are always welcome,
particularly when they come with some interface ideas as well!

\putbib[siunitx]
\end{bibunit}

%\bibliography{siunitx}
%\bibliographystyle{unsrtnat}
